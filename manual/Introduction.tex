\documentclass[12pt,twocolumn]{article}
\usepackage{xeCJK}%preamble part
\usepackage{graphicx}
\usepackage{indentfirst}
\usepackage[a4paper, inner=1.5cm, outer=3cm, top=2cm, bottom=3cm, bindingoffset=1cm]{geometry}
\usepackage{epstopdf}
\usepackage{array}
\usepackage{fontspec}
\usepackage{gensymb}
\usepackage[citecolor=blue]{hyperref}
\usepackage{amsmath}
\usepackage{makecell}
\usepackage[lofdepth,lotdepth]{subfig}
\setCJKmainfont[BoldFont={SimHei}]{SimSun}
\setCJKmonofont{SimSun}
\setmainfont{Times New Roman}
\newCJKfontfamily[hei]\heiti{SimHei}
\setlength{\extrarowheight}{4pt}
\setlength{\parindent}{1cm}
\begin{document}
\title{\textbf{\fontsize{15.75pt}{\baselineskip}{用户型德语初学者词典Klick Auf Deutsch Hilfer设计理念}}} 
\author{\fontsize{12pt}{\baselineskip}{计41 牛行知 数33 赵丰}}

\maketitle
\large
\twocolumn
\setlength{\columnseprule}{1pt}
\section{\textbf{\fontsize{12pt}{\baselineskip}{问题背景:我国双语学习型词典的设计缺位}}}
我国改革开放所引进或合作编纂出版的外来词典,其编纂设计者与使用者形成了主客二分的疏离关系,设计者对于需求的认知主要源于编者主体的专业知识判断,而非对实际用户需求调研的结果。\cite{Bib1}

我国原创英汉汉英类词典,在国内词典市场的份额缺失严重,这和脱离用户需求,盲目照翻单语词典不无关系。

从表面上看,我国电子词典呈现出一片繁荣景象,所涉语言从英汉延伸至其他小语种,但实际情况是电子产品公司与计算机软件开发公司对电子词典表现出极大的热情,辞书出版机构将纸制词典的电子版权转让给电子出版商,而后者只是简单地把印在纸上的东西搬进芯片。

欧美电子词典以辞书为本体来开发电子词典,注重词典数据库的建设,而我国内地则是以电子为本体,先是由IT公司开发,出现问题后又转向引进权威词典。\cite{Bib2}

\section{\textbf{\fontsize{12pt}{\baselineskip}{设计目标描述}}}
	电子词典的核心功能是单词查询,一般词典只能提供精确查询单词及变位入口点,我们试图在实现精确查询的基础上添加模糊查询(给出相关词)入口,通配符查询(用*表示任意字母)入口,短语查询入口等。
	
	电子词典的界面设计一方面要继承纸制词典的风格,另一方面在不同的媒介下有一定的发挥空间。

	为实现上述设计目标,我们以网页为平台开发了Klick Auf Deutsch Hilfer。
	
\section{\textbf{\fontsize{12pt}{\baselineskip}{设计理念综述}}}

\subsection{\textbf{\fontsize{12pt}{\baselineskip}{数据视角}}}
	我们设计词典抛弃了一般电子词典采用的急键值加表项的方式,而采用超文本标记语言xml组织每一个词项。单独的xml文件以词条的编号命名,如1.xml表示名词类中的第一个,词项是Abend;V100.xml表示动词类中的第100个,词项是wissen.
	
	xml可以自定义元素和属性,为此我们采用了文档类型定义dtd的方法,考虑到不同词性的词有不同的属性,每一个词性我们单独定义了一个dtd文件,在xml文档的头部,显示指明了它被哪一个dtd所约束,如NounModel.dtd。虽然不同词性的词有不同的dtd,但其大致结构相同。具体说来,每一个合法的xml首先都要有一个根元素名为Entry.Entry下必须依次出现元素Stichwort,Einheit/Anteil,zusammengesetzteWörter,
Synonymegruppe,Antonymegruppe,
Kollokationen,AllgemeineErläuterungen.

这里有的子元素结构比较简单,比如Stichwort下只包含了词形的信息。
Einheit/Anteil元素是适配考虑到清华大学德语教学正在使用的教材为每个单词提供的其所在单元和所在单元具体模块的信息。
zusammengesetzteWörter元素提供德语中的和该词有关的复合词的信息,
由于德语中复合词数量更多,相应的我们在zusammengesetzteWörter
下设了KompositaCollection和abgeleiteteWörter两个子元素,
分别包含合成词类和派生词类。在合成词类下,为支持后期多种检索方式,我们将其主要分为K\_和 \_K型,分别表示这个词项在该合成词的位置
是在前面还是后面,派生词类下对每个由该词项派生的词必须注明它的词性,否则按照dtd的语法检查规则,整篇文档就是不合法的。

对于Entry下接下来的三个元素,分别表示同义词集合、反义词集合和词组集合,其中我们在编辑的过程中发现,同反义词集合具有稀疏性。

整个词条中最重要的部分是AllgemeineErläuterungen,其结构也最复杂。考虑到一词多义的可能性,该“一般性释义”下设若干个Eintrag,
至少要有一个Eintrag.每一个Eintrag有Chinesisch和BeispielSammlung
两个子元素,分别是汉语释义和例句集,每一个例句集是由若干个
Beispiel组成的,而每一个Beipiel由Satz和Übersetzung组成。该部分对
整个文档树的深度贡献最大。

同时,我们考虑到单词之间错综复杂的语义关系,在Eintrag的相关子元素下设置了link属性,其值为相对应单词的文件名称,如essen词项下<\_K link=''1.xml''>Abendessen<\/K>就表示Abendessen可以链接到Abend.这种方法不仅为展示数据提供了统一个接口,还为用网络的方法做关联分析提供了数据基础。
\subsection{\textbf{\fontsize{12pt}{\baselineskip}{界面视角}}}

\begin{thebibliography}{}
\bibitem{Bib1}双语学习型词典设计特征研究 外研社2013年出版

\bibitem{Bib2} 计算词典学 上海辞书出版社2011年版

\bibitem{Bib3} \url{https://github.com/Leidenschaft/Deutsch-Lernen}
\end{thebibliography}
\end{document}